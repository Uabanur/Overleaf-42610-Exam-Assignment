
The logic behind the argument states, that the two models are mapped to their own paradigms by premise 2 and 3. The fourth premise states that the two paradigms have different elements in the disciplinary matrix. By accepting the second model, in \texttt{Claim 1}, the change from the first model to the second, means changing the elements in the disciplinary matrix which according to premise 1 cause the paradigm change. The paradigm change is therefore a valid consequence from the descriptive premises.

We will now go through the four premise, justifying them and try to create not just a valid argument, but also a sound argument.

First when we talk about what causes a paradigm change, we generally talk about a discrepancy between the norms/\textit{normal science} and the observed phenomena of the real world. If the anomalies are too great to be accounted for, by statistical error, a new paradigm is established, as a result of a crisis, to embrace and explain these new observations. If the new paradigm is accepted, a paradigm change occurs, forming a scientific revolution. The new paradigm itself is not sufficient to bring change; if it is not accepted, e.g. if the experiments were not trustworthy, the paradigm may be discarded. The implications of the paradigm change, is not only the scientific theories, but also how the world is perceived/the \textit{gestalt} would change, with the new perspective brought by the paradigm. This brings us to the first premise of the argument, by viewing the elements of the disciplinary matrix, as the cornerstones of the scientific norms, the mentioned discrepancies corresponds to a disagreement with the elements of the matrix. Changing the elements to explain the anomalies, is what characterizes the paradigm change, as mentioned in page 23 of the lecture 1 slides \textit{"The Paradigm as a disciplinary matrix"} \cite{lecture1slides}. Transitioning from a phase of pre-normal science to a new phase of normal science, where the implications of the paradigm is understood and explained convincingly.

The second and third premise, each bind the respective models, mentioned in the assignment description and the referenced article \cite{SolitonSite}, to their paradigms. The \textit{Hodgin-Huxley model} identifies the nerve pulses as dissipative pulses, such as electrical currents. The corresponding paradigm in neuroscience would have a \textit{shared model} for nerve pulses, visualizing the nerve pulse as an electrical current, together with \textit{exemplars} for the nerve pulses, i.e. predicting the propagation of the nerve pulse using theory from electromagnetism. 

The soliton model, identifies the nerve pulses as solitons which are non-dissipative singular waves, and therefore fundamentally different basis than the \textit{Hodgin-Huxley model}. A paradigm for the soliton model to operate in, would have a \textit{shared model} for nerve pulses, visualizing the nerve pulses as a kind of sound wave, propagating adiabatically through the nerves. This also leads to a new set of exemplars, much like with the \textit{Hodgin-Huxley model}, but here using the theory for solitons or sound waves.

The two models explain the observations fundamentally different, which is what leads to the different \textit{shared models} and \textit{exemplars}, and thereby the fourth premise. As both of these are elements of the disciplinary matrix (here what we call \textit{shared models} corresponds to what Kuhn calls \textit{metaphysical paradigms}), going from the \textit{Hodgin-Huxley model} to the \textit{soliton model}, would result in a change in these elements of the disciplinary matrix, which according to the first premise cause a paradigm change. 

This corresponds with the statement in \texttt{Claim 1}, and the conclusion of the initially stated argument, making it sound.