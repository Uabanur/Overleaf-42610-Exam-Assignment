When we talk about what causes a paradigm change, we generally talk about a discrepancy between the norms/\textit{normal science} and the observed phenomena of the real world. The normative procedures, as to how we expect the world to function, disagree with experiments. If the anomalies are too great to be accounted for, by statistical error, a new paradigm is established, as a result of a crisis, to embrace and explain these new observations. If the new paradigm is accepted, a paradigm shift/change occurs, forming a scientific revolution. It is important, that the new paradigm itself is not sufficient to bring change; if it is not accepted, e.g. if the experiments were not trustworthy, the paradigm may be discarded. The implications of the paradigm, is not only the scientific theories, but if it were accepted, how the world is perceived/the \textit{gestalt} would change, with the new perspective brought by the paradigm. This brings us to the first premise of the argument, by viewing the elements of the disciplinary matrix, as the cornerstones of the scientific norms, the mentioned discrepancies corresponds to a disagreement with the elements of the matrix. Changing the elements to explain the anomalies, is what characterizes the paradigm change, as mentioned in page 23 of the lecture 1 slides \textit{"The Paradigm as a disciplinary matrix"}. Transitioning from a phase of pre-normal science to a new phase of normal science, where the implications of the paradigm is understood and explained convincingly.

The second and third premise, each bind the respective models, mentioned in the assignment description and the referenced article \cite{SolitonSite}, to their paradigms. The \textit{Hodgin-Huxley model} identifies the nerve pulses as dissipative pulses, such as electrical currents. The corresponding paradigm in neuroscience would have a \textit{shared model} for nerve pulses, visualizing the nerve pulse as an electrical current, together with \textit{exemplars} for the nerve pulses, i.e. predicting the propagation of the nerve pulse using theory from electromagnetism. 

The soliton model, identifies the nerve pulses as solitons which is a non-dissipative singular wave, and therefore fundamentally different basis than the \textit{Hodgin-Huxley model}. A paradigm for the soliton model to operate in, would have a \textit{shared model} for nerve pulses, visualizing the nerve pulses as a kind of sound wave, propagating adiabatically through the nerves.


Solitons are known from quantum mechanics, as defined in \cite{book:949176} section 1.4, as:
\begin{center}
\textit{"...a localized wave that propagates along one space direction only, with undeformed shape ...a large- amplitude coherent pulse or very stable solitary wave, the exact solution of a wave equation, whose shape and speed are not altered by a collision with other solitary waves."}
\end{center}

Here the sound analogy corresponds to a near-soliton, as it is not confined to one direction in space, but sound waves may be a good example of the observed and confined pulses while being a more intuitive field compared to many quantum mechanical fields, and therefore ideal for \textit{exemplars}.

The two models explain the observations fundamentally different, which is what leads to the different \textit{shared models} and \textit{exemplars}, which brings us to the fourth premise. As both of these are elements of the disciplinary matrix (here what we call \textit{shared models} corresponds to what Kuhn calls \textit{metaphysical paradigms}), going from the \textit{Hodgin-Huxley model} to the \textit{soliton model}, would result in a change in these elements of the disciplinary matrix, which according to the first premise cause a paradigm change. This corresponds with the statement in \texttt{Claim 1}, and the conclusion of the initially stated argument.